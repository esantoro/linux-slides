\overlays{2} {
\begin{slide}{GNU/Linux... Ma che vuol dire?}

\onlySlide*{1} {
In queste pagine si parla di GNU/Linux... Ma non si chiamava "Linux" e
basta?
La risposta e no, ma anche si.
Linux, per essere precisi, indica il kernel, il \emph{cuore} del sistema
operativo... Quel pezzo di codice che sta a diretto contatto con il
ferro, ecco.

GNU, invece, è tutto il resto.

I programmi che usi, i driver, le interfacce grafiche, il browser,
l'editor... Quello è GNU.
}

\onlySlide*{2} {
\begin{itemize}
	\item[-] GNU sfrutta Linux per girare (dei programmi devono
	essere eseguiti da un kernel... se non c'è il kernel, dove
	girano i programmi?)
	\item[-] Linux sfrutta GNU per avere qualcosa da far girare (il
	kernel serve a far girare i programmi... se non ci sono
	programmi, a che server il kernel?)
\end{itemize}

{\huge GNU + Linux = GNU/Linux }
}

\end{slide} }
