\overlays{3}{
\begin{slide}{I pacchetti}

{\small
Il sistema è composto interamente da pacchetti. 

Tutte le moderne distrbuzioni hanno sistemi automatici per gestire
(scaricare e installare, rimuovere e aggiornare) i pacchetti.

\center{Che vuol dire?}

\onlySlide*{1}{
Vuol dire che se tu vuoi un pacchetto GNU/L si occupa di cercarlo per
te, di scaricarlo per te, di scompattarlo per te, di installarlo per te
e di configurarlo (basilarmente per te). Tenendo traccia di tutto quello
che hai installato.
}

\onlySlide*{2}{
Non vuoi più il pacchetto? GNU/Linux si occupa di andare a ricercare
tutti i files che ha installato un pacchetto per te, li rimuove per te e
disinstalla il pacchetto per te.
}

\onlySlide*{3}{
I tuoi pacchetti sono vecchi? GNU/Linux si occupa per te di vedere cosa
è vecchio e cosa no, scarica i pacchetti nuovi e per ogni pacchetto
elimina il vecchio e lo sostituisce con il nuovo. Tutte le
configurazioni, se non specificato, verranno mantenute!
}

}
\end{slide}
}
